\section{Related Work}

\subsection{Below the Surface}

The starting point for re-organizing the collection, creating aggregated datasets, and exploring overview data is the dataset downloadable in \textit{.csv} format on the Below the Surface project website. During the excavations around 700,000 objects were found, the dataset is a subset and contains around 20,000 objects that are digitally processed and rudimentary labeled. A separate data field description \footnote{https://belowthesurface.amsterdam/en/pagina/publicaties-en-datasets} file can be downloaded which further explains the controlled vocabulary used for the dataset.

\subsection{User Personas}

Defining the users is an important first step in developing a user focused data system \cite{Just-RightPersonas}.  As our team did not have direct access to potential users, we developed three user personas to help guide our decision making process.  

\subsubsection{Museum Curators}

Stephanie is a female museum curator, between the age of 35 to 45.  She has a masters education in fine art and spends her days curating exhibitions for the local history museum where she works.  She speaks English and Dutch.  Stephanie is good at networking and discovering new and interesting local history subjects.  She desires to find new stories, themes, and artifacts to be displayed in a local history museum in Amsterdam, The Netherlands.  Stephanie typically finds the things she is looking for by visiting other museums and exhibitions.  She also builds connections with other museum curators so that new exhibitions can be created.  Stephanie cannot spend a long time researching what artifacts could be included in an exhibit as she needs to deal with a lot of bureaucracy. As a result, she values quicker methods of find exhibit items.  Typically she uses a search engine, a museum website, news websites and face to face meetings as communication channels.

\subsubsection{Antique collectors}

Gerald is a male amateur antique collector, between the ages of 45 and 60.  He has a PhD in another field of study and is based in Europe.  He speaks English and has a growing collection of antiques from the 17th Century.  He is computer literate and has a desire to collect items which have some historic value, are from the 17th century, and have a unique visual appearance.  He collects his antiques in his spare time, so time spent searching for antiques is at a premium.  Gerald finds joy in hunting for antiques, especially using new ways to find antiques.  He see's optimisation of searching for antiques as a benefit.  His main channels fot searching are search engines, auction houses, ebay and antiques shops.

\subsubsection{Archaeologists}

Mădălina is a female archaeologist from Transylvania, Romania between the ages of 25 and 35.  She is currently in a PostDoc position as an archaeologist researching building materials of the 18th century in European nations that boarder the North Sea.  She speaks English and German.  Kate is interested in the functional use of certain artifacts from different time periods within the 18th century, specifically building materials and techniques. Kate wants to find materials that have a specific function related to buildings from the 18th century.  Kate prefers to physically visit materials, vist archives and museums.  She is involved in excavations of sites in major cities which are located around the North Sea coast.  She likes to spend time researching materials.  However, if she can make the search and selection of objects for viewing more efficient, she can spend more time analysing the finds that she does visit.  Kate's preferred channels for discovery are museum archives, search engines, and excavation project websites.

\subsection{Other websites}

Due to the user personas preferring websites similar to museum or archive websites, we searched for a number of others in the field to give us an idea.  We found 5 collection or museum archive websites which aided us in deciding how the website should be designed. We found that two sites had comparable layouts and search features found on the below the surface website \footnote{https://www.vangoghmuseum.nl/nl/collectie?q=} \footnote{https://www.guggenheim.org/collection-online}.  The other three websites had additional features which enriched the search capabilities for users. Such as, the ability to scroll down a collection based on the year \footnote{https://www.moma.org/calendar/exhibitions}, providing clear and simple visualisations corresponding to related data points \footnote{https://ourworldindata.org/how-many-animals-get-slaughtered-every-day}, and providing information on the numbers of tags and albums which have been highlighted by other users \footnote{https://www.fotozoektfamilie.nl}.  

As a result of the user personas we decided that the website needed to be designed in a similar way to other websites tin the field, give the ability to filter dataset based on material composition, to be a desktop based website, and available in English.