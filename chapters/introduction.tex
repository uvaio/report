\section{Introduction}

From 2003 to 2012 excavations took place for the creation of the North/South metro line in Amsterdam. At Damrak and Rokin, which are unlikely archaeological sites due to being in the city center, archaeologists had a chance to physically access the riverbed. During these excavations in the Amstel over 700,000 objects were preserved which resulted in the archaeological collection called 'Below the Surface' \footnote{https://belowthesurface.amsterdam/en} commissioned by the Municipality of Amsterdam. 

The collection has a great variety of objects, from tools over centuries old to credit cards recently lost which makes the collection a rare source of urban history. All objects are digitally processed (e.g. photographed, labeled, metadata added) and displayed on a front-end website at \textit{belowthesurface.amsterdam}. This website shows an overview of all the objects and a detail page with metadata of a particular object but no further categorization or classifications. This research aims to further organize this collection of objects with a focus on grouping the items by \textit{functional properties}, determining \textit{cultural relevance}, and researching \textit{object relationships}.